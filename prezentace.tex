\documentclass{beamer}
\usepackage{times}
\usepackage[czech]{babel}
\usepackage[utf8]{inputenc}
% \usetheme{Berkeley}
% \usetheme{Copenhagen}
\usetheme{Madrid}
% \usecolortheme{beaver}
\newcommand{\czuv}[1]{\quotedblbase #1\textquotedblleft}
\usepackage{graphicx}
\usepackage{amsmath}
\setbeamertemplate{caption}[numbered]

\title{Výkresy}
\subtitle{Tvorba ER diagramu}
\author{Míšová Miroslava a Matějka Jiří}
\date{\today}
%\logo{\includegraphics[width=3cm]{FIT.eps}}

\begin{document}
  \frame{\titlepage}
%%%%%%%%%%%%%%%%%%%%%%%%%%%%%%%%%%%%%%%%%%%%
  \begin{frame}
    \frametitle{Obsah}
    
    \begin{itemize}
      \item Zadání
      \item Entitní množiny
      \item Položky entitních množin
      \item Vztahy mezi entitními možinami
      \item Výsledný ER Diagram
    \end{itemize}
  \end{frame}
%%%%%%%%%%%%%%%%%%%%%%%%%%%%%%%%%%%%%%%%%%%%
  \begin{frame}
    \frametitle{Zadání - Výkresy}
    \hspace{2mm}
      \scriptsize{
      Předpokládejte, že analyzujete požadavky na systém, který bude poskytovat počítačovou 
      podporu pro kreslení stavebních výkresů. Z informací, které máte dosud k dispozici,
      vyplývá, že jednotkou, se kterou bude systém pracovat, bude výkres. Každý výkres bude
      mít svůj název, autora, datum poslední změny a řadu dalších atributů, a bude se týkat
      nemovitosti, u níž ukládáme název a místo.\par


      \hspace{0.5cm}Pro jednu nemovitost může existovat více výkresů. Výkres obsahuje jednu nebo více tzv.
      vrstev (např. vrstva s půdorysem budovy, rozvody plynu, elektřiny apod.), do nichž se
      umísťují geometrické útvary. Na jedné vrstvě se může nacházet řada geometrických útvarů,
      ale každý z nich je vždy jen na jedné vrstvě. Každá vrstva má své jednoznačné jméno.\par

  
      \hspace{0.5cm}Geometrické útvary lze rozdělit do dvou skupin -- primitivní a složené. Jako primitivní
      uvažujte bod, lomenou čáru a uzavřenou oblast. Složené útvary vznikají seskupením jiných
      útvarů (primitivních i složených). Protože musí být k dispozici i operace, která rozloží
      složený útvar na útvary, jejichž seskupením vznikl, musí být tato informace (tj. které
      prvky seskupení tvoří) k dispozici. Jedním z dalších požadavků je, aby systém pracoval
      s určitými rozšířitelnými paletami -- barev, typů čar a typů výplně. Každá vrstva má
      potom definovanou implicitní barvu (jedna barva  z  palety) a barvu podkladu, každý
      primitivní prvek může mít definovánu jinou barvu (opět z palety). Podobně lomená čára
      a uzavřená oblast mají definován typ čáry a uzavřená oblast navíc typ výplně – opět
      z příslušných palet.\par}

  \end{frame}
%%%%%%%%%%%%%%%%%%%%%%%%%%%%%%%%%%%%%%%%%%%%
  \begin{frame}
    \frametitle{Entitní množiny}
      \scriptsize{
      Předpokládejte, že analyzujete požadavky na systém, který bude poskytovat počítačovou 
      podporu pro kreslení stavebních výkresů. Z informací, které máte dosud k dispozici,
      vyplývá, že jednotkou, se kterou bude systém pracovat, bude \colorbox{blue!30}{výkres}. Každý výkres bude
      mít svůj název, autora, datum poslední změny a řadu dalších atributů, a bude se týkat
      \colorbox{blue!30}{nemovitosti}, u níž ukládáme název a místo.\par


      \hspace{0.5cm}Pro jednu nemovitost může existovat více výkresů. Výkres obsahuje jednu nebo více tzv.
      \colorbox{blue!30}{vrstev} (např. vrstva s půdorysem budovy, rozvody plynu, elektřiny apod.), do nichž se
      umísťují \colorbox{blue!30}{geometrické útvary}. Na jedné vrstvě se může nacházet řada geometrických útvarů,
      ale každý z nich je vždy jen na jedné vrstvě. Každá vrstva má své jednoznačné jméno.\par

  
      \hspace{0.5cm}Geometrické útvary lze rozdělit do dvou skupin -- \colorbox{blue!30}{primitivní} a \colorbox{blue!30}{složené}. Jako primitivní
      uvažujte bod, lomenou čáru a uzavřenou oblast. Složené útvary vznikají seskupením jiných
      útvarů (primitivních i složených). Protože musí být k dispozici i operace, která rozloží
      složený útvar na útvary, jejichž seskupením vznikl, musí být tato informace (tj. které
      prvky seskupení tvoří) k dispozici. Jedním z dalších požadavků je, aby systém pracoval
      s určitými rozšířitelnými paletami -- barev, typů čar a typů výplně. Každá vrstva má
      potom definovanou implicitní barvu (jedna barva  z  palety) a barvu podkladu, každý
      primitivní prvek může mít definovánu jinou barvu (opět z palety). Podobně lomená čára
      a uzavřená oblast mají definován typ čáry a uzavřená oblast navíc typ výplně – opět
      z příslušných palet.\par}
  \end{frame}
%%%%%%%%%%%%%%%%%%%%%%%%%%%%%%%%%%%%%%%%%%%%%
  \begin{frame}
    \frametitle{Vztahy mezi entitními množinami}
      \scriptsize{
      Předpokládejte, že analyzujete požadavky na systém, který bude poskytovat počítačovou 
      podporu pro kreslení stavebních výkresů. Z informací, které máte dosud k dispozici,
      vyplývá, že jednotkou, se kterou bude systém pracovat, bude výkres. Každý výkres bude
      mít svůj název, autora, datum poslední změny a řadu dalších atributů, a bude se týkat
      nemovitosti, u níž ukládáme název a místo.\par


      \hspace{0.5cm}Pro jednu nemovitost může existovat více výkresů. Výkres obsahuje jednu nebo více tzv.
      vrstev (např. vrstva s půdorysem budovy, rozvody plynu, elektřiny apod.), do nichž se
      umísťují geometrické útvary. Na jedné vrstvě se může nacházet řada geometrických útvarů,
      ale každý z nich je vždy jen na jedné vrstvě. Každá vrstva má své jednoznačné jméno.\par

  
      \hspace{0.5cm}Geometrické útvary lze rozdělit do dvou skupin -- primitivní a složené. Jako primitivní
      uvažujte bod, lomenou čáru a uzavřenou oblast. Složené útvary vznikají seskupením jiných
      útvarů (primitivních i složených). Protože musí být k dispozici i operace, která rozloží
      složený útvar na útvary, jejichž seskupením vznikl, musí být tato informace (tj. které
      prvky seskupení tvoří) k dispozici. Jedním z dalších požadavků je, aby systém pracoval
      s určitými rozšířitelnými paletami -- barev, typů čar a typů výplně. Každá vrstva má
      potom definovanou implicitní barvu (jedna barva  z  palety) a barvu podkladu, každý
      primitivní prvek může mít definovánu jinou barvu (opět z palety). Podobně lomená čára
      a uzavřená oblast mají definován typ čáry a uzavřená oblast navíc typ výplně – opět
      z příslušných palet.\par}
  \end{frame}
%%%%%%%%%%%%%%%%%%%%%%%%%%%%%%%%%%%%%%%%%%%%%%%%
  \begin{frame}
      \scriptsize{
      \frametitle{Výsledný ER diagram}
      Předpokládejte, že analyzujete požadavky na systém, který bude poskytovat počítačovou 
      podporu pro kreslení stavebních výkresů. Z informací, které máte dosud k dispozici,
      vyplývá, že jednotkou, se kterou bude systém pracovat, bude výkres. Každý výkres bude
      mít svůj název, autora, datum poslední změny a řadu dalších atributů, a bude se týkat
      nemovitosti, u níž ukládáme název a místo.\par


      \hspace{0.5cm}Pro jednu nemovitost může existovat více výkresů. Výkres obsahuje jednu nebo více tzv.
      vrstev (např. vrstva s půdorysem budovy, rozvody plynu, elektřiny apod.), do nichž se
      umísťují geometrické útvary. Na jedné vrstvě se může nacházet řada geometrických útvarů,
      ale každý z nich je vždy jen na jedné vrstvě. Každá vrstva má své jednoznačné jméno.\par

  
      \hspace{0.5cm}Geometrické útvary lze rozdělit do dvou skupin -- primitivní a složené. Jako primitivní
      uvažujte bod, lomenou čáru a uzavřenou oblast. Složené útvary vznikají seskupením jiných
      útvarů (primitivních i složených). Protože musí být k dispozici i operace, která rozloží
      složený útvar na útvary, jejichž seskupením vznikl, musí být tato informace (tj. které
      prvky seskupení tvoří) k dispozici. Jedním z dalších požadavků je, aby systém pracoval
      s určitými rozšířitelnými paletami -- barev, typů čar a typů výplně. Každá vrstva má
      potom definovanou implicitní barvu (jedna barva  z  palety) a barvu podkladu, každý
      primitivní prvek může mít definovánu jinou barvu (opět z palety). Podobně lomená čára
      a uzavřená oblast mají definován typ čáry a uzavřená oblast navíc typ výplně – opět
      z příslušných palet.\par}
  \end{frame}
%%%%%%%%%%%%%%%%%%%%%%%%%%%%%%%%%%%%%%%%%%%%%%%%%%%
% { % all template changes are local to this group.
%     \setbeamertemplate{navigation symbols}{}
%     \begin{frame}[plain]
%         \begin{tikzpicture}[remember picture,overlay]
%             \node[at=(current page.center)] {
%                 \includegraphics[width=\paperwidth]{yourimage}
%             };
%         \end{tikzpicture}
%      \end{frame}
% }
%%%%%%%%%%%%%%%%%%%%%%%%%%%%%%%%%%%%%%%%%%%%%%%%%%%%%%%%%%%%%%
  \begin{frame}
    \frametitle{Závěr}
    \begin{center}   
      Děkujeme za pozornost
    \end{center}
  \end{frame}
\end{document}

\documentclass[11pt, a4paper]{article}[]
	\usepackage[czech]{babel}
	\usepackage[utf8]{inputenc}
	\usepackage[left=2cm,text={17cm, 24cm},top=3cm, includefoot]{geometry}
	\usepackage{times}
	\usepackage{setspace}
	\usepackage{graphicx}
	\usepackage{lscape}

\begin{document}
	\thispagestyle{empty}

	\begin{center}
		\Huge
			\textsc{Vysoké učení technické v~Brně\\}
		\huge
			\textsc{Fakulta informačních technologií\\}
		\vspace{\stretch{0.382}}
		\LARGE
			IDS - Databázové systémy \\
		\Huge
			Dokumentace k projektu \\

		\LARGE	zadání projektu\\
			Restaurace \\
		\vspace{\stretch{0.618}}
	\end{center}
		{\Large \today \hfill Matějka Jiří (xmatej52), Míšová Miroslava (xmisov00)} % lze pouzit i \today
	\pagebreak
	\setcounter{page}{1}

	\section{Zadání}
	Vytvořte IS pro restaurační zařízení, který napomůže k zjednodušení a zpřehlednění jeho provozu. Restaurace je členěna do více místností a má přední a zadní zahrádku a poskytuje běžné stravovací služby veřejnosti. Od informačního systému se požaduje, aby, krom jiného, umožnil správu rezervací a objednávek. Rezervovat je možné jeden nebo více stolů v místnostech či na zahrádkách, anebo celé místnosti, případně i celou restauraci pro různé společenské akce. Součástí rezervace také může být objednávka nápojů a jídel. Systém musí umožňovat zaměstnancům restaurace vkládat objednávky spolu s informacemi, který zaměstnanec danou objednávku vložil a pro koho je určena. Když se zákazníci rozhodnou zaplatit, musí jim systém vystavit účtenku. Po zaplacení pak příslušný zaměstnanec vloží záznam o platbě do systému. Systém by měl také poskytovat podrobný přehled tržeb za vybrané období. Přístup k této funkci bude mít pouze majitel. V neposlední řadě musí systém evidovat veškeré prodávané jídlo a pití (včetně složení), přičemž majitel a odpovědný vedoucí mají možnost měnit ceny jídla a pití nebo přidávat a odebírat položky.
    \pagebreak
	\section{Konečný ER diagram}
	\begin{figure}[h]
	\begin{center}
		\includegraphics[height=18cm]{ERD.png}
		\label{fig:obr_ERD}
	\end{center}
	\end{figure}
    \pagebreak

    \section{Triggery}
    Oba vytvořené triggery pracují s tabulkou zaměstnanců. Jeden trigger generuje
    ID, druhý kontroluje platnost rodného čísla.

    \section{Procedury}
    Obě ze dvou procedur používají CURSOR a EXCEPTION. Procedura  \textit{procento\_vytvorenych\_rezervaci}
    má jako parametr ID zaměstnance a vypíše jeho jméno a kolik rezervací vytvořil
    (v procentech). Pracuje tedy se dvěmi tabulkami: \textit{Rezervace} a \textit{Zamestnanec}.
    Druhá procedura \textit{pocet\_zidli} má jako parametr lokaci a vypíše
    lokaci a počet židlí, které se v lokaci nacházejí. Pracuje tedy s tabulkou
    \textit{Stul}.

    \section{Přidělení práv uživateli xmisov00}
    Uživatel xmisov00 získal veškerá práva k procedurám a tabulkám, má tudíž
    oprávnění administrátora (např. majitel restaurace). Kuchař by měl, práva
    přístupu k tabulkám \textit{Surovina}, \textit{Potravina} a \textit{PObsahujeS}.

    \section{Materializovaný pohled}
    Spolu s materializovaným pohledem byly vytvořeny i logy pro něj (využití
    FAST REFRESH ON COMMIT - nemusí se spouštět celý dotaz materializovaného
    pohledu). Mimo REFRESH FAST ON COMMIT jsme do pohledu nastavili i CACHE,
    pro lepší načítání z pohledu, BUILD IMMEDIATE pro okamžité naplnění pohledu
    po vytvoření a ENABLE QUERY REWRITE pro zapnutí optimalizace.

    \section{INDEX a EXPLAIN PLAIN}
    EXPLAIN PLAIN byl vytvořen pro spojení 3 tabulek a to \textit{Objednavka}, \textit{Potravina}
    a \textit{OObsahujeP}. Z těchto tabulek byla vypočítána celková cena objednávek.
    Index byl vytvořen nad parametry tabulky \textit{OObsahujeP} a to nad ID objednávky
    a jménem potraviny. Po použití INDEX se procesorové zatížení značně zmenšilo
    (viz. další strana).
\pagebreak
    \begin{table}[h]
	\caption{Před použitím INDEX}
	\centering
	\begin{tabular}{|l|l|l|l|l|l|}
        \hline
        ID & Operation                        & Name            & Rows & Bytes & Cost (\% CPU) \\ \hline
        0  & SELECT STATEMENT                 &                 & 5    & 855   & 6 (17)        \\
        1  & --HASH GROUP BY                  &                 & 5    & 855   & 6 (17)        \\
        2  & ----NESTED LOOPS                 &                 & 5    & 855   & 6 (17)        \\
        3  & ------NESTED LOOPS               &                 & 5    & 855   & 6 (17)        \\
        4  & --------VIEWS                    & VW\_DBF\_5      & 5    & 460   & 4 (25)        \\
        5  & ----------HASH GROUP BY          &                 & 5    & 395   & 4 (25)        \\
        6  & ------------TABLE ACCES FULL     & OOBSAHUJEP      & 5    & 395   & 3 (0)         \\
        7  & --------INDEX UNIQUE SCAN        & SYS\_C001274290 & 1    &       & 0 (0)         \\
        8  & ------TABLE ACCES BY INDEX ROWID & POTRAVINA       & 1    & 79    & 1 (0)         \\ \hline
        \end{tabular}
        \end{table}

    \begin{table}[h]
	\caption{Po použití INDEX}
	\centering
	\begin{tabular}{|l|l|l|l|l|l|}
        \hline
        ID & Operation                        & Name              & Rows & Bytes & Cost (\% CPU) \\ \hline
        0  & SELECT STATEMENT                 &                   & 5    & 855   & 3 (0)         \\
        1  & --HASH GROUP BY                  &                   & 5    & 855   & 3 (0)         \\
        2  & ----NESTED LOOPS                 &                   & 5    & 855   & 3 (0)         \\
        3  & ------NESTED LOOPS               &                   & 5    & 855   & 3 (0)         \\
        4  & --------VIEWS                    & VW\_DBF\_5        & 5    & 460   & 1 (0)         \\
        5  & ----------HASH GROUP BY          &                   & 5    & 395   & 1 (0)         \\
        6  & ------------INDEX FULL SCAN      & INDEX\_OOBSAHUJEP & 5    & 395   & 1 (0)         \\
        7  & --------INDEX UNIQUE SCAN        & SYS\_C001274290   & 1    &       & 0 (0)         \\
        8  & ------TABLE ACCES BY INDEX ROWID & POTRAVINA         & 1    & 79    & 1 (0)         \\ \hline
        \end{tabular}
        \end{table}

    \subsection{Vysvětlení EXPLAIN PLAIN}
    Z tabulek je vidět, že se snížilo zatížení procesoru a též byl menší počet
    přístupů na disk. SELECT STATEMENT značí provedení samotného SELECT příkazu,
    HASH GROUP BY znamená, že se položky seskupí podle hashovacího klíče.
    Dva vnořené cykly ukazují průchod tabulkami. TABLE ACCES FULL značí průchod
    celou tabulkou bez použití indexu, INDEX FULL SCAN značí zase použití INDEXU.
    INDEX UNIQUE SCAN je přístup k tabulkám přes B-strom a nakonec TABLE ACCES BY INDEX ROWID
    je přístup do tabulky přes konkrétní řádek.

    \section{Závěr}
    Díky projektu jsme se seznámily s jazykem SQL a naučili se navrhovat databázové
    systémy. Pro vypracování projektu jsme používali nejen přednášky, ale hlavně
    externí zdroje, zejména fórum StackOverflow. Projekt jsme programovali v
    prostředí Oracle Developer, které nám bylo nabídno ve formě obrazu pro virtuální
    stroj. Dokumentaci jsme tvořili pomocí \LaTeX u.

\end{document}
